\documentclass[12pt, a4paper]{jsarticle}

\usepackage{geometry}
\geometry{
  top=30mm,
  bottom=30mm,
  left=25mm,
  right=25mm
}

\usepackage{fontspec}
\usepackage{zxjatype}
\usepackage{amsmath,amssymb}
\usepackage{graphicx}
\usepackage{float}
\usepackage{url}
\usepackage{setspace}
\usepackage{titlesec}
\usepackage{tocloft}
\usepackage{ragged2e}
\usepackage[hidelinks]{hyperref}

\onehalfspacing
\setlength{\parskip}{0.8em}

% Set URL line breaks
\def\UrlBreaks{\do\/\do-\do:\do=\do.\do?\do&\do_}
\Urlmuskip=0mu plus 1mu

% Set bibliography style
\renewcommand{\bibname}{参考文献}

\setmainfont[
  BoldFont=Hiragino Mincho ProN W6,
  ItalicFont=Hiragino Mincho ProN W3
]{Hiragino Mincho ProN W3}
\setjamainfont[
  BoldFont=Hiragino Mincho ProN W6,
  ItalicFont=Hiragino Mincho ProN W3
]{Hiragino Mincho ProN W3}
\setulmainfont[
  BoldFont=Hiragino Mincho ProN W6,
  ItalicFont=Hiragino Mincho ProN W3
]{Hiragino Mincho ProN W3}
\setmonofont{Osaka}
\setsansfont{Hiragino Kaku Gothic ProN W3}


\titleformat{\chapter}[hang]
  {\normalfont\Large\bfseries}{\thechapter.}{1em}{}
\titlespacing*{\chapter}{0pt}{20pt}{20pt}

\titleformat{\section}
  {\normalfont\large\bfseries}{\thesection}{1em}{}
\titlespacing*{\section}{0pt}{20pt}{10pt}

\titleformat{\subsection}
  {\normalfont\normalsize\bfseries}{\thesubsection}{1em}{}
\titlespacing*{\subsection}{0pt}{15pt}{8pt}

\renewcommand{\cfttoctitlefont}{\Large\bfseries\hfill}
\renewcommand{\cftaftertoctitle}{\hfill}
\renewcommand{\cftchapleader}{\cftdotfill{\cftdotsep}}
\renewcommand{\contentsname}{目次}

\title{\Huge 多言語対応RAGにおけるクロスリンガル性能の\\向上に関する研究}
\author{〇〇大学 〇〇学部 〇〇学科\\[10pt]
学籍番号: XXXXXXXX\\[5pt]
氏名: 〇〇 〇〇}
\date{\today}

\begin{document}

\maketitle
\thispagestyle{empty}
\newpage

\pagenumbering{roman}
\tableofcontents
\newpage

\pagenumbering{arabic}
\chapter{緒論}
\label{chap:introduction}

\section{研究の背景}
\label{sec:background}
% ここに研究背景を記述

\section{研究目的}
\label{sec:purpose}
% ここに研究目的を記述

\section{本論文の貢献}
\label{sec:contribution}
% ここに本論文の貢献を記述

\section{本論文の構成}
\label{sec:structure}
% ここに論文構成を記述

\chapter{関連研究}
\label{chap:related_work}

\section{Retrieval-Augmented Generation (RAG)}
\label{sec:rag_overview}
% RAGの基本について

\section{多言語・クロスリンガル情報検索}
\label{sec:multilingual_ir}
% 多言語検索、クロスリンガル検索について

\section{多言語RAGにおける既存アプローチ}
\label{sec:existing_multilingual_rag}
% Translate-Query, MultiRAG, CrossRAGなどの既存手法について

\subsection{Translate-Query RAG}
\label{subsec:translate_query}

\subsection{Multilingual RAG}
\label{subsec:multilingual_rag}

\subsection{CrossRAG}
\label{subsec:cross_rag}

\section{本研究の位置づけ}
\label{sec:positioning}
% 関連研究を踏まえた本研究の位置づけ

\chapter{提案手法}
\label{chap:proposed_method}

\section{システムアーキテクチャ概要}
\label{sec:architecture_overview}
% アーキテクチャ図をここに挿入

\section{デュアル検索モジュール}
\label{sec:dual_retrieval}

\section{証拠統一モジュール}
\label{sec:evidence_unification}
% 検索結果の翻訳と比較戦略について

\section{言語制御付き回答生成モジュール}
\label{sec:controlled_generation}
% プロンプトエンジニアリングによるコードスイッチング抑制について

\chapter{実証分析}
\label{chap:experiments}

\section{実験設定}
\label{sec:experimental_setup}

\subsection{データセット}
\label{subsec:datasets}
% MKQA, XOR-TyDi QAなど

\subsection{比較手法}
\label{subsec:baselines}
% ベースライン手法について

\subsection{評価指標}
\label{subsec:evaluation_metrics}
% EM, F1, コードスイッチング率など

\section{実験結果}
\label{sec:results}
% 結果をまとめた表などをここに挿入

\section{事例分析}
\label{sec:case_study}
% 成功例・失敗例の分析

\chapter{考察}
\label{chap:discussion}

\section{実験結果の解釈}
\label{sec:interpretation}
% なぜ提案手法が優れていたか

\section{本研究の限界と今後の課題}
\label{sec:limitations_future_work}

\chapter{結論}
\label{chap:conclusion}

\section{本研究のまとめ}
\label{sec:summary}

\section{今後の展望}
\label{sec:prospects}


\appendix
\chapter{謝辞}
\label{chap:acknowledgements}

\begin{thebibliography}{99}
  \raggedright
  % ここに参考文献を記述
  \bibitem{lewis2020rag}
  Lewis, P., et al. (2020). Retrieval-augmented generation for knowledge-intensive nlp tasks. \textit{Advances in Neural Information Processing Systems}, 33, 9459-9474.

\end{thebibliography}

\end{document}
